\documentclass[12pt]{article}
\usepackage[utf8]{inputenc}
\usepackage[english,russian]{babel}

\textheight=24.5cm % высота текста
\textwidth=16cm % ширина текста
\oddsidemargin=0pt % отступ от левого края
\topmargin=-1.5cm % отступ от верхнего края
\parindent=24pt % абзацный отступ
\parskip=0pt % интервал между абзацами
\tolerance=2000 % терпимость к "жидким" строкам
\flushbottom % выравнивание высоты страниц

\begin{document}
\thispagestyle{empty}
\begin{center}
\bigskip

\textbf{Отзыв на бакалаврскую диссертацию студента 4 курса\\
Грабового Андрея Валериевича\\
<<Анализ свойств локальных моделей в задачах кластеризации квазипериодических временных рядов>>}
\end{center}

В бакалаврской диссертации А.\,В. Грабового рассматривается задача построения признакового описания точек временного ряда для дальнейшей кластеризации точек данного ряда. В качестве признакового описания рассматривались главные компоненты сегмента фазовой траектории вблизи данной точки.

В работе предложен метод, который основывается на локальном снижении размерности фазовой траектории. В работе предложен метод построения признакового описания точек временного ряда, а также введена функция расстояния между различными точками в построенном пространстве признаков. При помощи предложенной функции расстояния выполнялась кластеризация точек временного ряда. После кластеризации точек временного ряда была произведена его сегментация внутри каждого кластера отдельно.

Также в работе был проведен вычислительный эксперимент, который показал, что предложенный алгоритм имеет хорошее качество кластеризации точек временного ряда. Также было показано, что в случае простой структуры ряда его сегментация происходит также с хорошим качеством.

За время выполнения работы А.\,В. Грабовой продемонстрировал способность самостоятельно решать поставленную задачу, а также творчески подходить к поиску ее решения. Следует отметить аккуратное и квалифицированное выполнение численных экспериментов и разработку программной системы, решающей поставленную задачу.

В течение обучения в бакалавриате А.\,В. Грабовым была опубликована работа <<Определение релевантности параметров нейросети>> в журнале <<Информатика и управление>>, а также были подготовлены ряд работ, которые готовы к публикации.

Работа является актуальным исследованием, удовлетворяет требованиям, предъявляемым к бакалаврским диссертациям в МФТИ, и заслуживает оценки <<отлично>>, а А.\,В. Грабовой~---~присвоения квалификации бакалавра и рекомендации в магистратуру.


\vspace{3cm}
\begin{flushleft}
Научный руководитель:\\
проф. каф. Интеллектуальные системы ФУПМ МФТИ, д.\,ф.-м.\,н.\\
\end{flushleft}
\begin{flushright}
В.\,В. Стрижов
\end{flushright}



\end{document}