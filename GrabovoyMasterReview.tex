\documentclass[12pt]{article}
\usepackage[utf8]{inputenc}
\usepackage[english,russian]{babel}

\textheight=24.5cm % высота текста
\textwidth=16cm % ширина текста
\oddsidemargin=0pt % отступ от левого края
\topmargin=-1.5cm % отступ от верхнего края
\parindent=24pt % абзацный отступ
\parskip=0pt % интервал между абзацами
\tolerance=2000 % терпимость к "жидким" строкам
\flushbottom % выравнивание высоты страниц

\begin{document}
\thispagestyle{empty}
\begin{center}
\bigskip

\textbf{Рецензия на магистерскую диссертацию студента 2 курса магистратуры\\
Грабового Андрея Валериевича\\
<<Обучение с экспертом для построения интерпретируемых моделей машинного обучения>>}
\end{center}

В магистерской диссертации А.\,В. Грабового рассматривается задача понижения пространства параметров моделей глубокого обучения на основе методов дистилляции и при помощи задания априорного распределения над параметрами аппроксимирующих моделей.

Одной из рассматриваемых задач в магистерской работе рассматривается дистилляция нейронных сетей. В работе проведен анализ классических методов дистилляции предложенных Дж. Хинтоном и В.\,Н. Вапником, которые используют ответы модели учителя для повышения качества модели ученика. Автором работы предложены методы обобщение на основе вероятностного подхода к дистиляции, а также при помощи байесовского вывода. Показано, что в этом случае при выполнении ряда условий начальные параметры ученика могут быть получены непосредственно из параметров учителя;

В рамках магистерской работы проведены вычислительные эксперименты, показывающие, что предложенный подход инициализации параметров дает лучшую сходимость при обучении.

Дипломная работа студента хорошо структурирована, изложение понятно. В работе есть большое число опечаток, но данные опечатки никак не мешают понимаю магистерской диссертации.

Работа является актуальным исследованием, удовлетворяет требованиям, предъявляемым к магистерским диссертациям и заслуживает оценки <<отлично>>, а А.\,В. Грабовой~---~присвоения квалификации магистра и рекомендации в аспирантуру.


\vspace{3cm}
\begin{flushleft}
Рецензент:\\
к.\,ф.-м.\,н.\\
\end{flushleft}
\begin{flushright}
А.\,А. Зайцев
\end{flushright}

\end{document}