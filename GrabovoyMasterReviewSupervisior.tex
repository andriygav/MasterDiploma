\documentclass[12pt]{article}
\usepackage[utf8]{inputenc}
\usepackage[english,russian]{babel}

\textheight=24.5cm % высота текста
\textwidth=16cm % ширина текста
\oddsidemargin=0pt % отступ от левого края
\topmargin=-1.5cm % отступ от верхнего края
\parindent=24pt % абзацный отступ
\parskip=0pt % интервал между абзацами
\tolerance=2000 % терпимость к "жидким" строкам
\flushbottom % выравнивание высоты страниц

\begin{document}
\thispagestyle{empty}
\begin{center}
\bigskip

\textbf{Отзыв на магистерскую диссертацию студента 2 курса магистратуры\\
Грабового Андрея Валериевича\\
<<Обучение с экспертом для построения интерпретируемых моделей машинного обучения>>}
\end{center}

В магистерской диссертации А.\,В. Грабового рассматривается задача понижения пространства параметров моделей глубокого обучения на основе методов дистилляции и методов задания априорного распределения над параметрами аппроксимирующих моделей.

В работе проведен анализ классических методов дистилляции предложеных Дж. Хинтоном и В.\,Н. Вапником, которые используют ответы модели учителя для повышения качества модели ученика.
В рамках проведенной работы студентом получено обобщение предложенного метода используя вероятностный подход к решению задачи.
Студентом предложен метод дистиляции использующий байесовский вывод. В рамках данного подхода апприорное распределение параметров ученика задается как функция от параметров учителя. В основе предложенного подхода используется идея сопоставления структур исходной модели учителя и структуры ученика. 

В рамках магистерской работы проведен вычислительный эксперимент, который показал, что предложенный алгоритм  позволяет улучшить качество ученика.

За время выполнения работы А.\,В. Грабовой продемонстрировал способность самостоятельно решать поставленную задачу, а также творчески подходить к поиску ее решения. Следует отметить аккуратное и квалифицированное выполнение численных экспериментов и разработку программной системы, решающей поставленную задачу.

В период обучения в бакалавриате и магистратуре А.\,В. Грабовым опубликованы работы <<Определение релевантности параметров нейросети>>, <<Введение отношения порядка на множестве параметров аппроксимирующих моделей>>, <<Quasi-periodic time series clustering for human>>, <<Анализ выбора априорного распределения для смеси экспертов>> в  в рецензируемых журналах, которые входят в список ВАК, а также были подготовлены ряд работ, которые готовы к публикации и находятся на этапе публикации.

В дополнение к исследуемым задачам А.\,В Грабовой является ассистентом в рамках курсов <<Прикладная статистика>> и <<Машинное обучение>>, что напрямую связано с его научной работой.

Работа является актуальным исследованием, удовлетворяет требованиям, предъявляемым к магистерским диссертациям в МФТИ, и заслуживает оценки <<отлично>>, а А.\,В. Грабовой~---~присвоения квалификации магистра и рекомендации в аспирантуру.


\vspace{3cm}
\begin{flushleft}
Научный руководитель:\\
проф. каф. Интеллектуальные системы ФУПМ МФТИ, д.\,ф.-м.\,н.\\
\end{flushleft}
\begin{flushright}
В.\,В. Стрижов
\end{flushright}

\end{document}