\documentclass[a4paper,14pt]{extarticle}
\usepackage[utf8]{inputenc}
\usepackage[english,russian]{babel}

\usepackage{amsthm}
\usepackage{graphicx}
\usepackage{caption}
\usepackage{amssymb}
\usepackage{amsmath}
\usepackage{mathrsfs}
\usepackage{euscript}
\usepackage{graphicx}
\usepackage{subfig}
\usepackage{caption}
\usepackage{color}
\usepackage{bm}
\usepackage{tabularx}
\usepackage{adjustbox}
\usepackage{url}
\usepackage{multirow}

\usepackage[toc,page]{appendix}

\usepackage{comment}
\usepackage{rotating}

\DeclareMathOperator*{\argmax}{arg\,max}
\DeclareMathOperator*{\argmin}{arg\,min}

\newtheorem{theorem}{Теорема}
\newtheorem{lemma}[theorem]{Лемма}
\newtheorem{definition}{Определение}[section]

\numberwithin{equation}{section}

\newcommand*{\No}{No.}

\begin{document}

% Титульный лист
\begin{titlepage}
    	\begin{center}
        		Министерство науки и высшего образования Российской Федерации Федеральное государственное автономное образовательное учреждение высшего образования
       
		«Московский физико-технический институт (национальный исследовательский институт)»

        		Физтех школа прикладной математики и информатики\\
        		Кафедра <<Интеллектуальные системы>>\\
    	\end{center}

    	\vspace{2cm}

    	\begin{center}
       		\Large \bf Выбор структуры моделей глубокого обучения
    	\end{center}
    
    	\begin{center}
		~\\[-28pt]
		Реферат к вступительному испытанию\\
		(Аспирантура)
	\end{center}
	
	\vspace{0.1cm}
	    
    \begin{center}
    	\textbf{Направление подготовки:} 09.06.01 Информатика и вычислительная техника
    \end{center}

   \vspace{0.1cm}

	\begin{flushright}
		\begin{table}[!ht]
			\centering
			\begin{tabular}{l}
				~~~~~~~~~~~~~~~~~~~~~~~~~~~~~~~~~~~~~~~~~~~~~~~~~~ \textbf{Выполнил:}\\
				~~~~~~~~~~~~~~~~~~~~~~~~~~~~~~~~~~~~~~~~~~~~~~~~~~ студент группы М05-904а\\
				~~~~~~~~~~~~~~~~~~~~~~~~~~~~~~~~~~~~~~~~~~~~~~~~~~ Грабовой Андрей Валериевич\\
				~~~~~~~~~~~~~~~~~~~~~~~~~~~~~~~~~~~~~~~~~~~~~~~~~~\\
				~~~~~~~~~~~~~~~~~~~~~~~~~~~~~~~~~~~~~~~~~~~~~~~~~~ \textbf{Научный руководитель:}\\
				~~~~~~~~~~~~~~~~~~~~~~~~~~~~~~~~~~~~~~~~~~~~~~~~~~ Доктор физико-математических наук\\
				~~~~~~~~~~~~~~~~~~~~~~~~~~~~~~~~~~~~~~~~~~~~~~~~~~ Стрижов Вадим Викторович\\
				~~~~~~~~~~~~~~~~~~~~~~~~~~~~~~~~~~~~~~~~~~~~~~~~~~ \\
	    	\end{tabular}
	    \end{table}
	\end{flushright}


	\begin{center}
		Москва 2021
	\end{center}

\end{titlepage}




% Нумерация должна начинаться со второй страницы
\setcounter{page}{2}

% Оглавление
\newpage
\tableofcontents

% Обозначения и сокращения
% \input{./dict.tex}

% Введение
\input{./introduction.tex}

% Основная часть
\input{./priorexpert.tex}
\input{./privlearn.tex}
\input{./bayesdistil.tex}
\newpage

\section{Аппроксимация кривых второго порядка при помощи обучения с экспертом}
\subsection{Описание задачи}
\label{intro}
Interpretable model building in machine learning~(Ribeiro et al. 2016) is one of the key challenges.
Modern solutions of the image classification problem based on deep learning networks ResNet, VGG, Intercept~(Kaiming et al. 2016) are poorly interpreted models.
The papers~(Han et al. 2020; Akhtar et al. 2018) show that deep learning networks are sensitive even to small noise in the data, which is due to their uninterpretability.

In this paper, we propose a \textit {training with an expert} method.
This method assumes the use of subject knowledge of experts to improve the quality of approximation, as well as to obtain interpretable machine learning models.
The subject knowledge of experts about the sample will be called~\textit {expert information}.
It is assumed that the use of expert information allows the sample to be approximated by simple interpretable models, such as linear models. Machine learning methods that take expert knowledge into account when building models are called~\textit {expert learning}.

This paper solves the problem of approximating second-order curves on a contour image. Second-order curves are selected for analysis, since they are easily described by linear models. In this case, these figures need to be restored in such applied problems as the problem of recognizing the iris of the eye~(Matveev 2010; Matveev et al. 2014; Bowyer et al. 2010), the problem of describing the particle track in the hadron collider~(Salamani et al. 2018). Expert information about a second-order curve allows you to map points on a plane into a new feature description, where each curve is approximated by one linear model. A model that approximates one curve is called a \textit {local model}. To approximate the entire contour image, it is required to approximate several second-order curves using several local models. In this paper, the following restrictions on images are introduced: a) the image consists only of second-order curves; b) the image is approximated by a small number of second-order curves; c) the number and type of curves in the image is known.

\begin{figure}[h!]
     \includegraphics[width=\textwidth]{results/priorexpertfig/explanation}
     \caption {Example: a) expert information of the first expert; b) baseline data; c) expert information of the second expert}
    \label{intro:fig2}
\end{figure}

Figure~\ref{intro:fig2} shows an example of second-order curves, as well as expert information on curves. Figure~\ref{intro:fig2} a shows the expert information of the first expert. Using this information, the first curve is fitted with a linear model and the second curve is noise. Figure~\ref{intro:fig2} b shows the expert information of the second expert. Using this information, the second curve is fitted with a linear model and the first curve is noise.

When approximating several curves on one contour image, a multi-model is built. An example of multi-models is a random forest~(Chen et al. 2012), tree boosting~(Chen et al. 2016), a mixture of experts~(Yuksel et al. 2012). In this paper, a mixture of experts is considered as a multi-model. Expert mixture is a multi-model that linearly weights local models that approximate a portion of the sample. The values of the weighting coefficients depend on the object for which the prediction is made. To solve the problem of a mixture of experts, a variational EM-algorithm~(Dempster et al. 1997; Bishop 2010; Peng et al. 1996) is used. The mixture of experts has many uses in a number of applications. In the paper~(Estabrooks et al. 2001), the text classification problem is solved. In the papers~(Cheung et al. 1995; Weigend et al. 2000; Cao 2003; Mossavat et al. 2010; Sminchisescu C et al. 2007; Tuerk 2001; Yumlu et al. 2003), a mixture of experts is used to predict time series for speech recognition, daily human activity, and prediction of the value of securities. In the paper~(Ebrahimpour et al. 2009), a mixture of experts was considered to solve the problem of recognizing handwritten numbers in images.

As an example, the problem of approximation of the iris image is considered. Figure~\ref{intro:fig1:real} shows an example of the image that needs to be approximated. In this paper, we consider a processed image, which is given in outline form, an example of such an image is shown in Figure~\ref{intro:fig1:outer}. Figure~\ref{intro:fig1:outer} shows two local circle models that approximate the iris of the eye. Circumferences are a simple example of a second order curve.

\begin{figure}[h!]
	\subfloat[]{\includegraphics[height = 0.2\textheight]{results/priorexpertfig/real_image}\label{intro:fig1:real}} 
	\subfloat[]{\includegraphics[height = 0.2\textheight]{results/priorexpertfig/outline_image}\label{intro:fig1:outer}} 
\caption{An example of the image of the iris of the eye and its outline representation: a) the image of the iris of the eye; b) contour image of the iris and approximating the given image of the circumferences}
\label{intro:fig1}
\end{figure}

For the problem of approximating the iris of the eye, the following expert information is used: the iris of the eye is approximated by two concentric circumferences. Expert information is used to construct a feature description of plane points, as well as to build an optimization function. The part of the error function for optimization that uses expert information is called a regularizer. Thus, the information that the image of the circumferences is specified by the feature description, and the information that the concentric circumferences are specified using a special regularizer.

In a computational experiment, the quality of the approximation of the contour image is analyzed depending on the specified expert information and on the noise level in the synthetically generated data. The analysis of the quality of the approximation of the iris is carried out, depending on the amount of expert information that was used to build the model. Note that each approximated image is a separate set of points that need to be approximated.

\subsection{Постановка задачи поиска параметров кривых второго порядка}
\label{sec:1}
Binary image is set:
$$ \mathbf{M} \in \{0, 1 \}^{m_1\times m_2},$$
where 1 corresponds to the black point of the image, and 0 corresponds to the white point of the background.
From the image $\mathbf {M} $, a sample$ \mathbf{C}$ is constructed, the elements of which are the coordinates $(x_i, y_i)$ of black points: $$\mathbf{C} \in \mathbb{R}^{N \times 2}. $$
The expert assumes that the image consists of a second-order curve~$\Omega$.
Let for a set of points $\mathbf {C} \in \mathbb {R}^{N \times 2} $ that form a curve $\Omega, $ expert information about the figure $E(\Omega) $is given.
The set $E (\Omega)$ consists of the shape $\Omega$ expected by the expert and the set of its admissible transformations. Based on the expert description, let us introduce mappings into a new problem for approximation:
\begin{equation}\label{eq1}
	K_{x}\bigl(E(\Omega)\bigr): \mathbb{R}^{2} \rightarrow \mathbb{R}^{n}, \quad K_{y}\bigl(E(\Omega)\bigr): \mathbb{R}^{2} \rightarrow \mathbb{R},
\end{equation} 
where~$K_{x}$ mapping objects to the attribute description of objects,~$n$is the number of features, and~$K_ {y}$ is a mapping to a target variable for an object. Applying the mappings~$K_ {x}, K_{y}$for the sample~$\mathbf {C}$ element by element we obtain:
\begin{equation}
\label{eq2}
	K_{x}\bigl(E(\Omega\bigr), \mathbf{c}) = \mathbf{x}, \quad  K_{y}\bigl(E(\Omega), \mathbf{c}\bigr) = y,
\end{equation}
where~$\mathbf{c} = (x_i, y_i)$ is a sample point $\mathbf{C}$.

Applying the mappings \eqref{eq2} to the original set of points $\mathbf {C} $, we obtain the sample
\begin{equation}
\label{eq4}
    \mathfrak{D} = \{(\mathbf{x}, y) \; | \; \forall \mathbf{c} \in \mathbf{C} \; \mathbf{x} = K_x(\mathbf{c}), \; y = K_y(\mathbf{c}) \}.
\end{equation}

We get that the original problem of curve approximation~$\Omega$ is reduced to approximation of the sample~$\mathfrak {D}$. In this paper, it is assumed that the sample~$\mathfrak {D}$ is approximated by a linear model:
\begin{equation}
	g(\mathbf{x}, \mathbf{w}) = \mathbf{x}^\mathsf{T} \mathbf{w},
\end{equation} 
where~$\mathbf{w}$ vector, the parameter to be found.

To find the optimal vector of parameters~$\hat {\mathbf {w}}$, it is required to solve the following optimization problem:
\begin{equation}
	\hat{\mathbf{w}} = \arg\min_{\mathbf{w}\in\mathbb{R}^n} \sum_{\left(\mathbf{x}, y\right) \in \mathfrak{D}}\|g(\mathbf{x}, \mathbf{w}) - y \|_2^2.
\end{equation} 

Thus, the problem of approximating the original curve~$\Omega$ is reduced to solving the problem of linear regression, i.e. finding the components of the vector $\hat{\mathbf {w}}$ connecting the resulting $\mathbf {x}$ and $y$.

In the case when on the image $K$ the second-order curves  $\Omega_1, \dots, \Omega_K$, for each of which there is expert information $E_k = E (\Omega_k), \, k \in \{1, \dots , K\}$, the problem of constructing a multi-model called a mixture of $K$ experts is posed.
\begin{definition}
We call the multimodel $ f $ a mixture of K experts
\begin{equation}
	f = \sum\limits_{k = 1}^{K}\pi_k(\mathbf{x}, \mathbf{V})g_k(\mathbf{w}_k),  \quad \pi_k(\mathbf{x}, \mathbf{V}): \mathbb{R}^{n\times |\mathbf{V}|} \rightarrow [0, \, 1], \quad \sum\limits_{k = 1}^{K}\pi_k(\mathbf{x}, \mathbf{V}) = 1, 
\end{equation}
where $g_k$ is a local model called by the expert, $\mathbf{x}$is an attribute description of an object, $\pi_k$is a gateway function, the vector $\mathbf{w}_k$ are local model parameters, the vector $\mathbf{V}$ are gateway function parameters. In this paper, $g_k$ is a linear model.
\end{definition}

For each second-order curve, mappings (\ref{eq1}) are given. For convenience, we introduce the following notation: $K_x^k\bigr(\mathbf{c}\bigr) = K_x\bigr(\Omega_k, \mathbf{c}\bigr)$ and $K_y^k\bigr(\mathbf{c}\bigr) = K_y\bigr(\Omega_k, \mathbf{c}\bigr)$. Then, using local linear models, we construct a universal multi-model describing the curves $\Omega_1, \dots, \Omega_K$ on the image $\mathbf{M}$:
\begin{equation}
\label{5}
	f = \sum\limits_{\mathbf{c} \in \mathbf{C}} \sum_{k = 1}^{K} \pi_k(\mathbf{c}, \mathbf{V})g_k(K^k_{x}\bigl(\mathbf{c}), \mathbf{w}_k), 
\end{equation}
where $\pi_k$ is the  gateway function. In this paper, we consider a simple case, where~$\mathbf{x}=K^1_{x}\bigl(\mathbf{c})=\cdots=K^K_{x}\bigl(\mathbf{c}),$ then the expression~\eqref{5} is rewritten in the following simple form:
\begin{equation}
\label{5_1}
	f = \sum\limits_{\mathbf{c} \in \mathbf{C}} \sum_{k = 1}^{K} \pi_k(\mathbf{x}, \mathbf{V})g_k(\mathbf{x}, \mathbf{w}_k), 
\end{equation}
where the gateway function $\pi_k$ has the following form:
\begin{equation}
\label{6}
	\pi_k(\mathbf{x}, \mathbf{V}): \mathbb{R}^{n\times |\mathbf{V}|} \rightarrow [0, \, 1], \; \; \; \; \sum\limits_{k = 1}^{K}\pi_k(\mathbf{x}, \mathbf{V}) = 1,
\end{equation}
where $\mathbf{V}$ are the gateway function parameters, and $g_k$ is a local model.
    
In this paper
\begin{equation}
    \boldsymbol{\pi}(\mathbf{x}, \mathbf{V}) = \text{softmax}\bigl(\mathbf{V}_1^{\mathsf{T}}\boldsymbol{\sigma}(\mathbf{V}_2^{\mathsf{T}}\mathbf{x}) \bigr),
\end{equation}
where $\mathbf{V} = \{ \mathbf{V}_1, \, \mathbf{V}_2\}$ are the gateway function parameters,
$\mathbf{V}_1 \in \mathbb{R}^{p \times k}, \, \mathbf{V}_2 \in \mathbb{R}^{n \times p}$. 

To find the optimal parameters of the multi-model, it is necessary to solve the following optimization problem:
\begin{equation}\label{9}
\mathcal{L} = \sum\limits_{(\mathbf{x}, y) \in \mathfrak{D}} \sum\limits_{k = 1}^{K} \pi_k(\mathbf{x}, \mathbf{V})(y - \mathbf{w}_k^{\mathsf{T}}\mathbf{x})^2 + R\bigl(\mathbf{V}, \mathbf{W}, E(\Omega)\bigr) \rightarrow \min_{\mathbf{V}, \mathbf{W}},
\end{equation}
where $\mathbf{W} = [\mathbf{w}_1, \dots, \mathbf{w}_k]$ are  parameters of local models, $R\bigl(\mathbf{V}, \mathbf{W}, E(\Omega)\bigr)$ is regularization parameters, based on expert information.

\subsection{Построение признакового описания фигур}
\label{sec:3}
\paragraph{ Unified space for second-order curves.} An arbitrary second-order curve, the main axis of which is not parallel to the ordinate axis, is given by the following expression:
\[
\label{st:coef}
x^2 = B'xy+C'y^2+D'x+E'y+F',
\]
where the coefficients $B ', C'$ are subject to restrictions that depend on the type of the curve. The expression~\eqref{eq2} takes the following form:
\[
\label{st:K_map}
K_x\bigr(\mathbf{c}_i\bigr)=\left[x_iy_i, y_i^2, x_i, y_i, 1\right], \quad K_y\bigr(\mathbf{c}_i\bigr)=x_i^2,
\]
whence we obtain the linear regression problem for recovering the parameters~$ B ', C', D ', E', F '$ from the selected sample.

\paragraph{ Circumference.} As a special case of a second-order curve, we consider the circumference.
Let $(x_0, y_0)$ be the center of the circumference to be found on the binary image $\mathbf {M} $, and $r$ be its radius.
The sample elements $(x_i, y_i) \in \mathbf {C}$ are the locus of points, which is approximated by the equation of the circumference:
\begin{equation}
(x_i - x_0)^2 + (y_i - y_0)^2 = r^2.
\end{equation}
Expanding the brackets, we get:
\begin{equation}(2x_0)\cdot x_i + (2y_0)\cdot y_i + (r^2 - x_0^2 - y_0^2)\cdot 1 = x_i^2 + y_i^2 . 
\end{equation}
Then the presentations (\ref{eq2}) take the following form:
\begin{equation}
\label{10}
K_{x}(\mathbf{c}_i) = [x_i, \, y_i, \, 1] = \mathbf{x}, \,  K_{y}(\mathbf{c}_i) = x_i^2+y_i^2 = y.
\end{equation} 
Assign the linear regression problem \eqref{eq4}.
Vector components $\mathbf{w} = [w_0, \, w_1, \, w_2]^\mathsf{T}$, binding $\mathbf{x}$ and $y$, restore the parameters of the circumference: \begin{equation} x_0 = \frac{w_0}{2}, \; y_0 = \frac{w_1}{2}, \; r = \sqrt{w_3 + x_0^2 + y_0 ^2}.\end{equation}

\subsection{Композиция фигур}
\label{sec:4}
o construct a composition of figures, we will use the expression~\eqref{9}, which takes the following form:
\begin{equation} 
\label{statment:optim:task}
\begin{aligned}
\mathcal{L} = \sum\limits_{\mathbf{c} \in \mathbf{C}} \sum\limits_{k = 1}^{K} \pi_k(\mathbf{c}, \mathbf{V})\left(K^{k}_y\bigr(\mathbf{c}\bigr) - \mathbf{w}_k^{\mathsf{T}}K^{k}_x\bigr(\mathbf{c}\bigr)\right)^2 + R\bigl(\mathbf{V}, \mathbf{W}, E(\Omega)\bigr) \rightarrow \min_{\mathbf{V}, \mathbf{W}},
\end{aligned}
\end{equation} 
where~$K^{k}_x, K^{k}_y$ expert representation of the~$k$-th expert. Assuming that all curves in the image are described by one attribute description $\mathbf {x} =K^{1}_x\bigr(\mathbf{c}\bigr)=\cdots=K^{K}_x\bigr(\mathbf{c}\bigr), x= K^{1}_y\bigr(\mathbf{c}\bigr)=\cdots=K^{K}_y\bigr(\mathbf{c}\bigr),$ we get the following optimization problem:
\begin{equation} 
\label{statment:optim:task:simp}
\begin{aligned}
\mathcal{L} = \sum\limits_{\left(\mathbf{x}, y\right) \in \mathfrak{D}} \sum\limits_{k = 1}^{K} \pi_k(\mathbf{x}, \mathbf{V})\left(y - \mathbf{w}_k^{\mathsf{T}}\mathbf{x}\right)^2 + R\bigl(\mathbf{V}, \mathbf{W}, E(\Omega)\bigr) \rightarrow \min_{\mathbf{V}, \mathbf{W}},
\end{aligned}
\end{equation} 

As a regularizer~$R$, additional restrictions on the vectors of model parameters are considered. To solve the optimization problem~\eqref{statment:optim:task:simp} it is proposed to use the EM-algorithm.

\subsection{Вычислительный эксперимент}
\label{sec:5}

A computational experiment was carried out to analyze the quality of models of second-order curves in the image. The experiment is divided into several parts. The first part is an experiment with several circumferences in the image. The second part analyzes the convergence of the method depending on the noise level in the data and on the specified expert information. In the third part, an experiment is conducted to approximate the iris of the eye.

\begin{figure}[h!]
	\subfloat[]{\includegraphics[height = 0.2\textheight]{results/priorexpertfig/900.eps}} 
	\subfloat[]{\includegraphics[height = 0.2\textheight]{results/priorexpertfig/901.eps}}
	\subfloat[]{\includegraphics[height = 0.2\textheight]{results/priorexpertfig/902.eps}} 

\caption{Multi-model depending on different prior assumptions and noise level. From left to right: circumferences without noise; noise in the radius of the circle; noise in the radius of a circle as well as arbitrary points throughout the image.}
\label{ce:fig3}
\end{figure}

\begin{figure}[h]
	\subfloat[]{\includegraphics[height = 0.2\textheight]{results/priorexpertfig/900noise.eps}} 
	\subfloat[]{\includegraphics[height = 0.2\textheight]{results/priorexpertfig/901noise.eps}}\\
	\subfloat[]{\includegraphics[height = 0.2\textheight]{results/priorexpertfig/902noise.eps}} 

\caption{Dependence of the parameters  $r$, $x_0$ and $y_0$ on the iteration number for different prior distributions. From left to right: circumferences without noise; noise in the radius of the circle; noise in the radius of a circle as well as arbitrary points throughout the image.}
\label{ce:fig4}
\end{figure}

In this part of the experiment, an example of training a multi-model is shown to approximate several second-order figures simultaneously. A synthetic sample is used as data, which is obtained by generating three arbitrary non-intersecting circumferences, as well as adding noise to these circumferences. Noise was added to the radius of the circle for each point, and random points were added to the sample that do not belong to circumstances.

Figure~\ref{ce:fig3} shows the result of building an ensemble of locally approximating models that approximate the sample. Each local model approximates one circumference, and when adding different noise, the quality of the approximation will drop.
Figure~\ref{ce:fig4} shows a graph of the dependence of the radius of the circumferences $r$ and their centers $(x_0, y_0)$ on the iteration number. 

\begin{figure}[h!t]
\includegraphics[width=0.8\textwidth]{results/priorexpertfig/beta_gamma}
\caption{ The result of the approximation for data with different noise levels~$\beta$ and on the variance of the prior distribution~$\gamma$}
\label{ce:fig6}
\end{figure}

\begin{figure}[h!t]
\includegraphics[width=0.5\textwidth]{results/priorexpertfig/3dplot}
\caption{Dependence of models on the noise level~$\beta$ in the data, as well as on the variance of the prior distribution~$\gamma$}
\label{ce:fig5}
\end{figure}
In this part of the experiment, we analyze the quality of approximation~$S$ on the noise level~$\beta$ in the data and on the parameter of a priori distributions~$\gamma$. The sample is obtained as followsfirst, two vectors of parameters are randomly selected~$\mathbf{w}^\text{true}_{1}$ and~$\mathbf{w}^\text{true}_{2}$ are  coefficients of two parabolas. The vectors~$\mathbf {w}^\text {true}_{1} $ and~$\mathbf{w}^\text {true}_{2} $ are used to generate points~$x_i$ and~$y_i$ with normal noise added~$\varepsilon\sim\mathcal {N} \bigr(0,\beta\bigr)$. When training a multi-model, the prior distribution of parameters is considered~$\mathbf{w}_1\sim\mathcal{N}\bigr(\mathbf{w}^\text{true}_{1}, \gamma\mathbf{I}\bigr),\mathbf{w}_2\sim\mathcal{N}\bigr(\mathbf{w}^\text{true}_{2}, \gamma\mathbf{I}\bigr)$.

The following quality criterion is considered:
\[
S = ||\mathbf{w}^\text{pred}_{1} - \mathbf{w}^\text{true}_{1}||^{2}_{2} + ||\mathbf{w}^\text{pred}_{2} - \mathbf{w}^\text{true}_{2}||^{2}_{2},
\]
where~$\mathbf{w}^\text{pred}_{1}$ approximation of the vector of parameters of the first local model, and~$\mathbf{w}^\text{pred}_{2}$ approximation of the vector of parameters of the second local model.

Figure~\ref{ce:fig5} shows the dependence of the quality criterion~$S$ on the noise level~$\beta$ and the a priori distribution parameter~$\gamma$. The graph shows that at a low noise level~$\beta$ the quality of the approximation does not depend on the parameter~$\gamma $, and with an increase in the noise~$\beta $ the quality of the approximation~$S$ decreases.

Figure~\ref{ce:fig5} shows an example of how the algorithm works with different parameters~$\beta$ and $\gamma$. It is seen that in the absence of noise~$\beta$, both local models approximate the sample. With an increase in the noise level, the quality of the approximation decreases: at~$\beta = 0 {,}2$, with an increase in $\gamma$, the first local model from a parabola goes over to an ellipse; for~$\beta=0 {,}4$ as $\gamma$ increases, the first local model from a parabola goes over to an ellipse, and the second model from a parabola goes over to a hyperbola.

\begin{figure}[h!]
	\subfloat[]{\includegraphics[height = 0.17\textheight]{results/priorexpertfig/not_prior_real_example}} 
	\subfloat[]{\includegraphics[height = 0.17\textheight]{results/priorexpertfig/prior_real_example}}
	\subfloat[]{\includegraphics[height = 0.17\textheight]{results/priorexpertfig/prior_regular_real_example}} 
\caption{Visualization of the approximation of the iris: a) if the $ R_0 $ regularizer is specified; b) if the $ R_1 $ regularizer is specified; b) if the $ R_2 $ regularizer is specified}
\label{ce:fig6}
\end{figure}


\begin{figure}
     \includegraphics[width=\textwidth]{results/priorexpertfig/experiment_real_not_prior}\\
     \caption{Visualization of the multi-model convergence process in the case of a regularizer~$R_0$}
    \label{ce:fig7}
\end{figure}

\begin{figure}
     \includegraphics[width=\textwidth]{results/priorexpertfig/experiment_real_prior}
     \caption{Visualization of the multi-model convergence process in the case of a regularizer~$R_1$}
    \label{ce:fig8}
\end{figure}

\begin{figure}
     \includegraphics[width=\textwidth]{results/priorexpertfig/experiment_real_regular}
     \caption{Visualization of the multi-model convergence process in the case of a regularizer~$R_2$}
    \label{ce:fig9}
\end{figure}

An analysis of the quality of the approximation is carried out for the problem of approximating the iris of the eye in the image. The iris of the eye consists of two concentric circumferences, therefore, a multi-model is considered, which consists of two experts: each expert approximates one of the circumstances. In a computational experiment, the quality of the approximation of circumferences is compared in the case of specifying different regularizers $R_0, R_1, R_2$. Regularizer$R_0\bigl(\mathbf{V}, \mathbf{W}, E(\Omega)\bigr)=0,$ that is, there is no regularizer. Regularizer:
\[
R_1\bigl(\mathbf{V}, \mathbf{W}, E(\Omega)\bigr)= -\sum_{k=1}^{K}\mathbf{w}_k^{\mathsf{T}}\mathbf{w}_k,
\]
which romotes near-zero parameters of local models.
Regularizer 
\[
R_2\bigl(\mathbf{V}, \mathbf{W}, E(\Omega)\bigr)= -\sum_{k=1}^{K}\mathbf{w}_k^{\mathsf{T}}\mathbf{w}_k + \sum_{k=1}^{K}\sum_{k'=1}^{K}\sum_{j=1}^2\left(w_k^j-w_k'^j\right)^2,\]
which promotes the coincidence of the centers of the circumferences and close to zero parameters of the model.
Figure~\ref{ce:fig6} shows the result of the eye iris approximation algorithm after 10 iterations. It can be seen that in the absence of a regularizer, one of the circumferences is found incorrectly. If the regularizer~$R_1 $ is given, the model approximates both circumferences with good quality, but the circumferences are not concentric. In case of specifying the regularizer~$R_2$, we get concentric circumferences on the image.

Figure~\ref{ce:fig7}--\ref{ce:fig9} shows the process of convergence of multi-models in the case of specifying different regularizers~$R_0, R_1, R_2$. It can be seen that the models with the regularizer type~$R_1$ and~$R_2$ approximate both circumferences, and the multi-model with the$R_0$ regularizer approximates only the large circumference.

\newpage

\section{Введение отношения порядка на множестве параметров аппроксимирующих моделей}
\subsection{Описание задачи}
В данной работе предлагается метод введения отношения порядка на множестве параметров сложных параметрических моделей, таких как нейросеть. Рассматривается порядок, заданный при помощи ковариационной матрицы градиентов функции ошибки по параметрам модели~\cite{Mandt2017}. В работе~\cite{Chunyan2016} предложен итерационный метод для поиска ковариационной матрицы градиентов. Данный итерационный метод интегрируется в градиентный метод оптимизации Adam~\cite{Kingma2014}.

Множество параметров упорядочивается по возрастанию дисперсии: от параметра с минимальной дисперсией до параметра с максимальной дисперсией градиента функции ошибки по соответствующему параметру модели. Предполагается, что малая дисперсия градиента указывает на то, что соответствующий параметр можно зафиксировать.

Для задания порядка на множестве параметров при помощи ковариационной матрицы вводится предположение о том, что фиксация параметров происходит в момент, когда все параметры модели находятся в некоторой окрестности локального минимума функции ошибки. Данное условие накладывается для корректного использования итерационного метода поиска ковариационной матрицы градиентов.

Заданный порядок на множестве параметров модели используется для фиксации тех параметров модели, которые оказываются предстоящими с точки зрения заданного порядка. Сначала фиксируются те параметры, которые имеют минимальную дисперсию градиента в окрестности локального минимума функции ошибки.

Для анализа свойств предложенного метода задания порядка на множестве параметров проводился вычислительный эксперимент. В качестве моделей рассматривались модели различной структурной сложности: линейные модели, нейросетевые модели. Предложенный метод задания порядка сравнивается с методом, в котором порядок задан произвольным образом.
\subsection{Формальная постановка задачи}
Задана выборка:
\begin{equation}
\label{eq:st:1}
\begin{aligned}
\mathfrak{D} = \bigr\{\bigr(\textbf{x}_i, y_i\bigr)\bigr\}_{i=1}^{m}, \quad \textbf{x}_{i} \in \mathbb{X} = \mathbb{R}^{n}, \quad y_i \in \mathbb{Y},
\end{aligned}
\end{equation}
где $n$~---~размерность признакового пространства, $m$~---~число объектов в выборке. Пространство ответов~$\mathbb{Y} = \mathbb{R}$ в случае задачи регрессии и  $\mathbb{Y} = \{1,\cdots, K\}$ в случае задачи классификации, где~$K$~---~число классов.

Задано семейство моделей параметрических функций с наперед заданной структурой:
\begin{equation}
\label{eq:st:2}
\begin{aligned}
\mathfrak{F} &= \bigr\{f\bigr(\textbf{w}\bigr):\mathbb{X} \to \mathbb{Y}~| \textbf{w} \in \mathbb{R}^{p}\bigr\}, \\ 
\mathbf{h}\bigr(\textbf{w}, \textbf{x}\bigr) &= \textbf{W}_1\bm{\sigma}\bigr(\textbf{W}_2\bm{\sigma}\bigr(\cdots\bm{\sigma}\bigr(\textbf{W}_r\textbf{x}\bigr)\cdots\bigr)\bigr),\\
f_{\text{\text{cl}}}\bigr(\textbf{w}, \textbf{x}\bigr) &= \arg \max_{j \in \bigr\{1,\cdots, K\bigr\}} \text{softmax}\bigr(\mathbf{h}\bigr(\textbf{w}, \textbf{x}\bigr)\bigr)_{j}, \\ 
f_{\text{reg}}\bigr(\textbf{w}, \textbf{x}\bigr) & = \mathbf{h}\bigr(\textbf{w}, \textbf{x}\bigr), 
\end{aligned}
\end{equation}
где $p$~---~размерность пространства параметров,~$r$~---~число слоев нейросети,~$\textbf{w} = \text{vec}[\textbf{W}_1, \textbf{W}_2, \cdots, \textbf{W}_r]$, а $\bm{\sigma}$~---~функция активации. В случае задачи регрессии структура модели имеет вид~$f_{\text{\text{reg}}}$, а в случае классификации имеет вид~$f_{\text{\text{cl}}}$.
%В качестве $\tau$ рассматривается~$\tau\bigr(\textbf{x}\bigr)~=~\textbf{x}$ в случае задачи регрессии, в случае задачи многоклассовой классификации~$\bm{\sigma}\bigr(\textbf{x}\bigr)~=~\text{softmax}\bigr(\textbf{x}\bigr)$.
Задана функция потерь:
\begin{equation}
\label{eq:st:3}
\begin{aligned}
\mathcal{L}\bigr(\textbf{w}, \mathfrak{D}\bigr) &= \frac{1}{m}\sum_{i=1}^{m}l\bigr(\textbf{x}_{i}, y_i, \textbf{w}\bigr),\\
l_{\text{\text{reg}}}\bigr(\textbf{x}, y, \textbf{w}\bigr) &= \bigr(y - f\bigr(\textbf{w}, \textbf{x}\bigr)\bigr)^{2},\\
l_{\text{\text{cl}}}\bigr(\textbf{x}, y, \textbf{w}\bigr) &= -\sum_{j=1}^{K}\bigr([y = j]\ln\text{softmax}_j\bigr(\mathbf{h}\bigr(\textbf{w}, \textbf{x}\bigr)\bigr)\bigr),
\end{aligned}
\end{equation}
где $l_{\text{\text{reg}}}$~---~это функция ошибки на одном элементе для задачи регрессии, $l_{\text{\text{cl}}}$~---~для задачи классификации.
Оптимальный вектор параметров $\hat{\textbf{w}}$ получим минимизацией функции потерь:
\begin{equation}
\label{eq:st:0:1}
\begin{aligned}
\hat{\textbf{w}} = \arg \min_{\textbf{w}\in\mathbb{R}^{p}} \mathcal{L}\bigr(\textbf{w}, \mathfrak{D}\bigr).
\end{aligned}
\end{equation}

 \subsection{Задание отношение порядка на множестве параметров}
Для поиска оптимальных параметров модели используется градиентный метод оптимизации:
\begin{equation}
\label{eq:st:4}
\begin{aligned}
\textbf{w}_{t} = \textbf{w}_{t-1} + \Delta\textbf{w}\bigr(\textbf{g}_{S,t}, \textbf{w}_{t-1}, \textbf{w}_{t-2}, \cdots\bigr), \quad \textbf{g}_{S,t}=\frac{\partial \mathcal{L}\bigr(\textbf{w}_{t}, \textbf{X}_{S}, \textbf{Y}_{S}\bigr)}{\partial \textbf{w}},
\end{aligned}
\end{equation}
где $t$~---~номер итерации, $\textbf{g}_{S,t}$~---~значение градиента на подвыборке размера $S$, $\Delta\textbf{w}$~---~приращение вектора параметров.
 
 
Порядок на множестве параметров модели задается при помощи ковариационной матрицы~$\textbf{C}$ градиентов функции ошибки~$\mathcal{L}$ по параметрам модели~$\textbf{w}$. Для вычисления ковариационной матрицы $\textbf{C}$ используется итерационная формула~\cite{Chunyan2016}, которая вычисляется на каждой итерации~\eqref{eq:st:4} градиентного метода оптимизации параметров:
\begin{equation}
\label{eq:st:5}
\begin{aligned}
\textbf{C}_t = \bigr(1-\kappa_t\bigr)\textbf{C}_{t-1}+\kappa_t\bigr(\textbf{g}_{1,t}-\textbf{g}_{S,t}\bigr)\bigr(\textbf{g}_{1,t}-\textbf{g}_{S,t}\bigr)^{\mathsf{T}},
\end{aligned}
\end{equation}
 где $t$~---~номер итерации, $\textbf{g}_{S,t}$~---~значение градиента на подвыборке размера $S$, $\textbf{g}_{1,t}$~---~значение градиента на первом элементе подвыборки, $\kappa_t=\frac{1}{t}$~---~параметр сглаживания, $\textbf{C}_0$ инициализируются из равномерного распределения.
 
Пусть известно $t_0$~---~число итераций, после которого все параметры находятся в некоторой локальной окрестности минимума, тогда, как показано в работе~\cite{Chunyan2016}, матрица~$\textbf{C}_{t_0}$ аппроксимирует истинную ковариационную матрицу~$\textbf{C}$. Ковариационная матрица $\textbf{C}_{t_0}$ используется для упорядочения параметров модели $\textbf{w}_{t_0}$. 
 
Пусть $\mathcal{I}$~---~ упорядоченный вектор индексов $[1, 2, \cdots, p]$. Обозначим $\mathcal{I}_{\textbf{w}_{t_0}}$ вектор индексов, порядок которого задан при помощи ковариационной матрицы $\textbf{C}_{t_0}$. 
 
Например, если ковариационная матрица $\textbf{C}_{t_0}$  имеет вид
 $$
\begin{bmatrix}
0{,}3& 0 & 0\\
0& 0{,}2 & 0\\
0& 0 & 0{,}25\\
\end{bmatrix},
 $$
 то вектор индексов $\mathcal{I}_{\textbf{w}_{t_0}} = [3,1,2]$.
 

\subsection{Фиксация параметров}
Для фиксации параметров $\textbf{w}_{t_0}$ при помощи вектора индексов $\mathcal{I}_{\textbf{w}_{t_0}}$ используется бинарный вектор $\bm{\alpha}\bigr(k\bigr)$:
\begin{equation}
\label{eq:st:6}
\begin{aligned}
\alpha_i\bigr(k\bigr) = \begin{cases}
   1, &\text{если }\mathcal{I}_{\textbf{w}_{t_0}}[j] \leq k;\\
   0 &\text{иначе},
 \end{cases}
\end{aligned}
\end{equation}
 где $k$~---~число фиксирующих параметров.
 
 Учитывая~\eqref{eq:st:6}, уравнение~\eqref{eq:st:4} приводится к виду
 \begin{equation}
\label{eq:st:7}
\begin{aligned}
\textbf{w}_{t} = \textbf{w}_{t-1} + \bm{\alpha}\bigr(k\bigr)\cdot\Delta\textbf{w}\bigr(\textbf{g}_{S,t}, \textbf{w}_{t-1}, \textbf{w}_{t-2}, \cdots\bigr),
\end{aligned}
\end{equation}
где $t$~---~номер итерации, $\textbf{g}_{S,t}$~---~значение градиента на подвыборке размера $S$, $\Delta\textbf{w}$~---~приращение вектора параметров. После умножения на бинарный вектор $\bm\alpha$ часть параметров не оптимизируется, что приводит к фиксации параметров.

\newpage

\section{Автоматическое определение релевантности параметров}
\subsection{Описание задачи}
Данная работа посвящена прореживанию структуры сети. Предлагается удалять наименее релевантные параметры модели. Под релевантностью~\cite{cun1990} подразумевается то, насколько параметр влияет на функцию ошибки. Малая релевантность указывает на то, что удаление этого параметра не влечет значимого изменения функции ошибки. Метод предлагает построение исходной избыточной сложности нейросети с большим количеством избыточных параметров. Для определения релевантности параметров предлагается оптимизацировать параметры и гиперпараметры в единой процедуре. Для удаления параметров предлагается использовать метод Белсли~\cite{neychev2016}.

\subsection{Постановка задачи}

Задана выборка
$$\mathfrak{D} = \{\textbf{x}_i,y_i\},~ i =1,...,N, \eqno(2.1)$$
где~$\textbf{x}_i \in \mathbb{R}^{m}$,~$y_i \in \{1, \dots, Y\}$,~$Y$ --- число классов.
Рассмотрим модель~$f(\mathbf{x}, \mathbf{w}): \mathbb{R}^m \times \mathbb{R}^n \to \{1,\dots,Y\}$, где~$\textbf{w} \in \mathbb{R}^n$ --- пространство параметров модели,

$$f(\mathbf{x}, \mathbf{w}) = \text{softmax}\bigl( f_1(f_2(...(f_l(\mathbf{x}, \mathbf{w})\bigr), \eqno(2.2)$$
где~$f_i(\mathbf{x}, \mathbf{w}) =  \text{tanh}(\mathbf{w}^\mathsf{T}\mathbf{x})$,~$l$ --- число слоев нейронной сети,~$i \in \{1\dots l\}$.
Параметр~$w_j$ модели~$f$  называется активным, если~$w_j \not = 0$. Множество индексов активных параметров обозначим~$\mathcal{A} \subset \mathcal{J} = \{1,...,n\}$.
Задано пространство параметров модели:
$$\mathbb{W_\mathcal{A}} = \{ \textbf{w} \in \mathbb{R}^n~|~w_j\not=0,~j \in \mathcal{A}  \}, \eqno(2.3)$$


Для модели~$f$ с множеством индексов активных параметров~$\mathcal{A}$ и соответствующего ей вектора параметров~$\textbf{w} \in \mathbb{W_\mathcal{A}}$  определим логарифмическую функцию правдоподобия выборки:
$$\mathcal{L}_\mathfrak{D}(\mathfrak{D}, \mathcal{A}, \textbf{w}) = \log p(\mathfrak{D}|\mathcal{A}, \textbf{w}), \eqno(2.4)$$
где~$p(\mathfrak{D}|\mathcal{A},\textbf{w})$ --- апостериорная вероятность выборки~$\mathfrak{D}$ при заданных~$\textbf{w}, \mathcal{A}$.
Оптимальные значения~$\textbf{w},\mathcal{A}$ находятся из минимизации~$-\mathcal{L}_{\mathcal{A}}(\mathfrak{D},\mathcal{A})$ --- логарифма правдоподобия модели:
$$\mathcal{L}_{\mathcal{A}}(\mathfrak{D},\mathcal{A}) =\log p(\mathfrak{D}|\mathcal{A}) = \log  \int_{{\textbf{w}\in\mathbb{W_\mathcal{J}}}}
p(\mathfrak{D} | \textbf{w}) p(\textbf{w} | \mathcal{A}) d \textbf{w}, \eqno(2.5)$$
где~$p(\textbf{w}|\mathcal{A})$ ---  априорная вероятность вектора параметров в пространстве~$\mathbb{W_\mathcal{J}}$.

Так как вычисление интеграла (2.5) является вычислительно сложной задачей, рассмотрим вариационный подход~\cite{bishop2006} для решения этой задачи. Пусть задано распределение~$q$:
$$q(\textbf{w})\sim \mathcal{N}(\textbf{m}, \textbf{A}^{-1}_\text{ps}), \eqno(2.6)$$
где~$\textbf{m}, \textbf{A}^{-1}_\text{ps}$ --- вектор средних и матрица ковариации, аппроксимирующее неизвестное апостериорное распределение~$p(\textbf{w}|\mathfrak{D},\mathcal{A})$:
$$p(\textbf{w} | \mathcal{A})\sim \mathcal{N}(\boldsymbol{\mu},\textbf{A}^{-1}_{\text{pr}}), \eqno(2.7)$$
где~$\boldsymbol{\mu},\textbf{A}^{-1}_{\text{pr}}$ --- вектор средних и матрица ковариации.

Приблизим интеграл (2.5) методом предложеном в \cite{bishop2006}:
$$\mathcal{L}_{\mathcal{A}}(\mathfrak{D},\mathcal{A}) = \log p(\mathfrak{D}|\mathcal{A}) = $$
$$ =\int_{\textbf{w}\in\mathbb{W_\mathcal{J}}} q(\textbf{w}) \log \frac{p(\mathfrak{D}, \textbf{w}|\mathcal{A})}{q(\textbf{w})}d \textbf{w} - \int_{\textbf{w}\in\mathbb{W_\mathcal{J}}}  q(\textbf{w}) \log \frac{p(\textbf{w}|\mathfrak{D},\mathcal{A})}{q(\textbf{w})}d \textbf{w} \approx $$
$$\approx \int_{\textbf{w}\in\mathbb{W_\mathcal{J}}} q(\textbf{w}) \log \frac{p(\mathfrak{D}, \textbf{w}|\mathcal{A})}{q(\textbf{w})}d \textbf{w} = $$
$$= \int_{\textbf{w}\in\mathbb{W_\mathcal{J}}} q(\textbf{w}) \log \frac{p(\textbf{w}| \mathcal{A})}{q(\textbf{w})}d \textbf{w} + \int_{\textbf{w}\in\mathbb{W_\mathcal{J}}} q(\textbf{w}) \log p(\mathfrak{D}|\mathcal{A}, \textbf{w})d \textbf{w}=$$

$$=\mathcal{L}_\textbf{w}(\mathfrak{D}, \mathcal{A}, \textbf{w})+\mathcal{L}_{E}(\mathfrak{D},\mathcal{A}). \eqno(2.8)$$

Первое слагаемое формулы (2.8) --- это сложность модели. Оно определяется расстоянием Кульбака-Лейблера:
$$\mathcal{L}_\textbf{w}(\mathfrak{D}, \mathcal{A}, \textbf{w}) = -D_{KL}\bigl(q(\textbf{w})||p(\textbf{w}|\mathcal{A})\bigr). \eqno(2.9)$$
Второе слагаемое формулы (2.8) является матожиданием правдоподобия выборки~$\mathcal{L}_\mathfrak{D}(\mathfrak{D},\mathcal{A}, \textbf{w})$. В данной работе оно является функцией ошибки:
$$\mathcal{L}_{E}(\mathfrak{D},\mathcal{A}) = \mathsf{E}_{\textbf{w}\sim q}\mathcal{L}_\mathfrak{D}(\textbf{y}, \mathfrak{D}, \mathcal{A}, \textbf{w}). \eqno(2.10)$$

Требуется найти параметры, доcтавляющие минимум суммарному функционалу потерь~$\mathcal{L}_\mathcal{A}(\mathfrak{D},\mathcal{A},\textbf{w})$ из (2.8):
$$\hat{\textbf{w}} = \argmin_{\mathcal{A}\subset\mathcal{J},~\textbf{w} \in \mathbb{W_\mathcal{A}}} -\mathcal{L}_\mathcal{A}(\mathfrak{D}, \mathcal{A}, \textbf{w}) = $$
$$=\argmin_{\mathcal{A}\subset\mathcal{J},~\textbf{w} \in \mathbb{W_\mathcal{A}}} D_{KL}\bigl(q(\textbf{w})||p(\textbf{w}|\mathcal{A})\bigr) - \mathcal{L}_\mathfrak{D}(\mathfrak{D}, \mathcal{A}, \textbf{w}). \eqno(2.11)$$

\subsection{Случайное удаление}
Метод случайного удаления заключается в том, что случайным образом удаляется некоторый параметр $w_\xi$ из множества активных параметров сети.  Индекс параметра $\xi$ из равномерного распределения  случайная величина, предположительно доставляющая оптимум в (2.11).
$$\xi \sim \mathcal{U}(\mathcal{A}). \eqno(3.1.1)$$

\subsection{Оптимальное прореживание}
Метод оптимального прореживания \cite{cun1990} использует вторую производную целевой функции (2.4) по параметрам для определения нерелевантных параметров. Рассмотрим функцию потерь~$\mathcal{L}$ (2.4) разложенную в ряд Тейлора в некоторой окрестности вектора параметров~$\textbf{w}$:
$$\delta \mathcal{L} = \sum_{j\in \mathcal{A}} g_j\delta w_j + \frac{1}{2}\sum_{i,j\in \mathcal{A}} h_{ij}\delta w_i\delta w_j + O(||\delta\textbf{w}||^3), \eqno(3.2.1)$$
где~$\delta w_j~$ --- компоненты вектора~$\delta\textbf{w}$,~$g_j$ --- компоненты вектора градиента~$\nabla \mathcal{L}$, а~$h_{ij}$ --- компоненты гесcиана~$\textbf{H}$:
$$g_j = \frac{\partial \mathcal{L}}{\partial w_j}, \qquad h_{ij} = \frac{\partial^2\mathcal{L}}{\partial w_i \partial w_j}. \eqno(3.2.2)$$

Задача является вычислительно сложной в силу размерности матрицы \textbf{H}. Введем следующее предположение \cite{cun1990}, о том что удаление нескольких параметров приводит к такому же изменению функции потерь~$\mathcal{L}$, как и суммарное изменение при индивидуальном удалении:
$$\delta \mathcal{L} = \sum_{j\in \mathcal{A}} \delta \mathcal{L}_j, \eqno(3.2.3)$$
где~$\mathcal{A}$ --- множество активных параметров,~$\delta\mathcal{L}_j$ --- изменение функции потерь, при удалении одного параметра~$\textbf{w}_j$.

В силу данного предположения будем рассматривать только диагональные элементы матрицы \textbf{H}. После введенного предположения, выражение (3.2.1) принимает вид
$$\delta \mathcal{L} = \frac{1}{2} \sum_{j\in \mathcal{A}} h_{jj}\delta w_j^2, \eqno(3.2.4)$$

Получаем следующую задачу оптимизации:

$$\xi = \argmin_{j\in \mathcal{A}} h_{jj}\frac{w_j^2}{2}, \eqno(3.2.5)$$
где~$\xi$ --- индекс наименее релевантного, удаляемого параметра, предположительно доставляющая оптимум в (2.11).

\subsection{Удаление неинформативных параметров с помощью вариационного вывода}
Для удаления параметров в работе \cite{graves2011} предлагается удалить параметры, которые имеют максимальное отношение плотности~$p(\textbf{w}|\mathcal{A})$ априорной вероятности в нуле к плотности вероятности априорной вероятности в математическом ожидании параметра.\\
Для гауссовского распределения с диагональной матрицей ковариации получаем:
$$p_j(\textbf{w}|\mathcal{A})(x) = \frac{1}{\sqrt[]{2\sigma_j^2}}\exp({-\frac{(x-\mu_j)^2}{2\sigma_j^2}}). \eqno(3.3.1)$$
Разделив плотность вероятности в нуле к плотности в математическом ожидание
$$ \frac{p_j(\textbf{w}|\mathcal{A})(0)}{p_j(\textbf{w}|\mathcal{A})(\mu_j)}= \exp({-\frac{\mu_j^2}{2\sigma_j^2}}), \eqno(3.3.2)$$
Получаем следующую задачу оптимизации:
$$\xi = \argmin_{j\in \mathcal{A}} \left|\frac{\mu_j}{\sigma_j}\right|, \eqno(3.3.3)$$
где~$\xi$ --- индекс наименее релевантного, удаляемого параметра.

\subsection{Прореживание сети на основе метода Белсли}
Предлагается метод основанный на модификации метода Белсли. Пусть $\textbf{w}$ --- вектор параметров доставляющий минимум функционалу потерь $\mathcal{L}$ на  множестве $\mathbb{W_\mathcal{A}}$, а $\textbf{A}_\text{ps}$ соответствующая ему ковариационная матрица.

Выполним сингулярное разложение матрицы
$$\textbf{A}_\text{ps} = \textbf{U}{\bf\Lambda}\textbf{V}^\mathsf{T}. \eqno(4.1)$$
Индекс обусловленности $\eta_{j}$ определим как отношение максимального элемента к $j$-му элементу матрицы ${\bf\Lambda}$. Для нахождения мультиколлиниарных признаков требуется найти индекс~$\xi$ вида:
$$\xi = \argmax_{j\in \mathcal{A}}{\eta_j}. \eqno(4.2)$$

\begin{figure}[h!t]\center
\subfloat[Матрица ковариации]{\includegraphics[width=0.5\textwidth]{results/Cov.pdf}}
\subfloat[Дисперсионные доли]{\includegraphics[width=0.5\textwidth]{results/BelslyImage.pdf}}
\caption{Илюстрация метода Белсли}
\label{CovBel}
\end{figure}

\begin{table}[h]
\begin{center}
\caption{Илюстрация метода Белсли}
\begin{tabular}{|c|cccccc|}
\hline
$\eta$ & $q_1$& $q_2$& $q_3$& $q_4$& $q_5$& $q_6$\\
\hline
$1.0$ &  $2\cdot 10^{-17}$ &  $4\cdot 10^{-17}$ &  $1\cdot 10^{-16}$ &  $2\cdot 10^{-17}$ &  $6\cdot 10^{-17}$&  $3\cdot 10^{-4}$ \\
\hline
$1.5$ &  $5\cdot 10^{-17}$ &  $9\cdot 10^{-17}$ &  $2\cdot 10^{-16}$ &  $5\cdot 10^{-17}$ &  $3\cdot 10^{-20}$ &  $3\cdot 10^{-2}$ \\
\hline
$3.3$ &  $9\cdot 10^{-18}$ &  $1\cdot 10^{-17}$ &  $2\cdot 10^{-17}$ &  $9\cdot 10^{-18}$ &  $2\cdot 10^{-19}$ &  $9\cdot 10^{-1}$ \\
\hline
$2\cdot 10^{15}$ &  $1\cdot 10^{-2}$ &  $1\cdot 10^{-1}$ &  $8\cdot 10^{-1}$ &  $2\cdot 10^{-3}$ &  $9\cdot 10^{-2}$ &  $1\cdot 10^{17}$ \\ 
\hline
$8\cdot 10^{15}$ &  $6\cdot 10^{-2}$ &  $8\cdot 10^{-1}$ &  $9\cdot 10^{-2}$ &  $8\cdot 10^{-2}$ &  $9\cdot 10^{-1}$ & $ 2\cdot 10^{17} $\\
\hline
$1\cdot 10^{16}$ &  $\bf9\cdot 10^{-1}$ &  $1\cdot 10^{-2}$& $ 4\cdot 10^{-2}$&  $\bf9\cdot 10^{-1}$ &  $1\cdot 10^{-3}$ & $ 5\cdot 10^{-21}$ \\
\hline
\end{tabular}
\label{CovBelTable}
\end{center}
\end{table}

Дисперсионный долевой коэффициент $q_{ij}$ определим как вклад $j$-го признака в дисперсию $i$-го элемента вектора параметра~$\textbf{w}$:

$$q_{ij} = \frac{u^2_{ij}/\lambda_{jj}}{\sum^n_{j=1}{u^2_{ij}/\lambda_{jj}}}. \eqno(4.3)$$

Большие значение дисперсионных долей указывают на наличие зависимости между параметрами. Находим долевые коэффициенты, которые вносят максимальный вклад в дисперсию параметра~$w_\xi$:

$$\zeta = \argmax_{j\in \mathcal{A}}{q_{\xi j}}. \eqno(4.4)$$
Параметр с индексом $\zeta$ определим как наименее релевантный параметр нейросети. 
%Для удаления нескольких зависимых параметров за раз, предлагается удалить параметры с номерами $p \in \mathcal{A}: q_{\xi i} > \lambda  \in  \mathbb{R}_+$.

Проиллюстрируем принцип работы метода Белсли на примере. Рассмотрим данные порожденные следующим образом: 
$$\textbf{w} = \begin{bmatrix}
\text{sin}(x)\\
\text{cos}(x)\\
\text{2+cos}(x)\\
\text{2+sin}(x)\\
\text{cos}(x) + \text{sin}(x)\\
x
\end{bmatrix}$$
с матрицей ковариации на рис.~\ref{CovBel}.a, где $x \in [0.0, 0.02, ..., 20.0]$.


В табл.~\ref{CovBelTable} приведены индексы обусловленности и соответствующие им дисперсионные доли, которые также изображены на рис.~\ref{CovBel}.b. Согласно этим данным, максимальный индекс обусловленности $\eta_6 = 1.2\cdot 10^{16}$. Ему соответствуют максимальные дисперсионные доли признаков с индексами 1 и 4, которые, как видно из построения выборки, являются линейно зависимые.

\input{./samplesize.tex}
\input{./series.tex}

% Выводы
\input{./conclusion.tex}

% Библиографические ссылки
\newpage


\begin{thebibliography}{10}
\bibitem{cifar10}
	\textit{Alex Krizhevsky and Vinod Nair and Geoffrey Hinton} CIFAR-10 (Canadian Institute for Advanced Research) // \url{http://www.cs.toronto.edu/~kriz/cifar.html}
\bibitem{imagenet}
	\textit{Deng, J., Dong, W., Socher, R., Li, L.-J., Li, K., Fei-Fei, L. } Imagenet: A large-scale hierarchical image database //  IEEE conference on computer vision and pattern recognition, 2009. P. 248--255. 
	
\bibitem{Zehao2017}
	\textit{{Huang}, Zehao and {Wang}, Naiyan} Like What You Like: Knowledge Distill via Neuron Selectivity Transfer // arXiv e-prints, 2017.
\bibitem{Zheng2020}
	\textit{Kui Ren and Tianhang Zheng and Zhan Qin and Xue Liu} Adversarial Attacks and Defenses in Deep Learning // Engineering, 2020. P. 346--360.
\bibitem{Krizhevsky2012}
	\textit{Alex Krizhevsky, Ilya Sutskever, Geoffrey Hinton} ImageNet Classification with Depp Convolutional Neural Networks // NIPS, 2012.
\bibitem{Simonyan2014}
	\textit{Karen Simonyan and Andrew Zisserman} Very Deep Convolutional Networks for Large-Scale Image Recognition // NIPS, 2014.
\bibitem{Vaswani2017}
	\textit{Vaswani A., Shazeer N., Parmar N., Uszkoreit J., Jones L., Gomez A., Kaiser L., Polosukhin I.} Attention Is All You Need // In Advances in Neural Information Processing Systems. 2017. V. 5. P. 6000--6010.
\bibitem{Devlin2018}
       \textit{Devlin J., Chang M., Lee K., Toutanova K.} BERT: Pre-training of Deep Bidirectional Transformers for Language Understanding // arXiv preprinted, 2018.
\bibitem{Brown2020}
        \textit{Tom B. Brown et al} GPT3: Language Models are Few-Shot Learners // arXiv preprinted, 2020.
\bibitem{Linting2021}
        \textit{Linting Xue and Noah Constant and Adam Roberts and Mihir Kale and Rami Al-Rfou and Aditya Siddhant and Aditya Barua and Colin Raffel.} mT5: A massively multilingual pre-trained text-to-text transformer // arXiv preprinted, 2021.
\bibitem{Ziqing2020}
        \textit{Yang, Ziqing and Cui, Yiming and Chen, Zhipeng and Che, Wanxiang and Liu, Ting and Wang, Shijin and Hu, Guoping} {T}ext{B}rewer: {A}n {O}pen-{S}ource {K}nowledge {D}istillation {T}oolkit for {N}atural {L}anguage {P}rocessing // Proceedings of the 58th Annual Meeting of the Association for Computational Linguistics: System Demonstrations.  2020. P. 9--16.
\bibitem{Kaiming2015}
	\textit{He K., Zhang X., Ren S., Sun J.} Deep Residual Learning for Image Recognition // Proc. of the IEEE Conference on Computer Vision and Pattern Recognition. Las Vegas, 2016. P. 770--778.
\bibitem{bachteev2018}
	\textit{Бахтеев О.\,Ю., Стрижов В.\,В.} Выбор моделей глубокого обучения субоптимальной сложности // АиТ. 2018. № 8. С. 129--147.
\bibitem{Hinton2015}
        \textit{Hinton G., Vinyals O., Dean J.} Distilling the Knowledge in a Neural Network // NIPS Deep Learning and Representation Learning Workshop. 2015.
\bibitem{mnist}
	\textit{LeCun Y.,  Cortes C., Burges C.} The MNIST dataset of handwritten digits, 1998. \text{http://yann.lecun.com/exdb/mnist/index.html}.
\bibitem{Vapnik2015}
	\textit{Vapnik V., Izmailov R.} Learning Using Privileged Information: Similarity Control and Knowledge Transfer // Journal of Machine Learning Research. 2015. No 16. P. 2023--2049.
\bibitem{Lopez2016}
	\textit{Lopez-Paz D., Bottou L., Scholkopf B., Vapnik V.} Unifying Distillation and Privileged Information // In International Conference on Learning Representations. Puerto Rico, 2016.
\bibitem{Ivakhnenko1994}
	\textit{Madala H., Ivakhnenko A.} Inductive Learning Algorithms for Complex Systems Modeling. Boca Raton: CRC Press Inc., 1994.
\bibitem{fashionmnist}
	\textit{Xiao H., Rasul K.,Vollgraf R.} Fashion-MNIST: a Novel Image Dataset for Benchmarking Machine Learning Algorithms // arXiv preprint arXiv:1708.07747. 2017.
\bibitem{twiter2013}
	\textit{Wilson T., Kozareva Z., Nakov P., Rosenthal S., Stoyanov V., Ritter A.} {S}em{E}val-2013 Task 2: Sentiment Analysis in Twitter // Proceedings of the Seventh International Workshop on Semantic Evaluation ({S}em{E}val 2013). Atlanta, 2013. P. 312--320.
\bibitem{LeCun1989}
	\textit{LeCun Y., Boser B., Denker J., Henderson D., Howard R., Hubbard W., Jackel L.} Backpropagation Applied to Handwritten Zip Code Recognition // Neural Computation. 1989. V. 1. No 4. P. 541--551.
\bibitem{Schmidhuber1997}
	\textit{Hochreiter S., Schmidhuber J.} Long short-term memory // Neural Computation. 1997. V. 9. No 8.  P. 1735--1780.
\bibitem{kingma2014}
	\textit{Kingma D, Ba J.} Adam: A Method for Stochastic Optimization // arXiv preprint arXiv:1412.6980. 2014.
\bibitem{graves2011}
	\textit{Graves A.} Practical Variational Inference for Neural Networks // Advances in Neural Information Processing Systems, 2011. Vol. 24. P. 2348--2356.
\bibitem{Vapnik2015}
	\textit{Vapnik V., Izmailov R.} Learning Using Privileged Information: Similarity Control and Knowledge Transfer // Journal of Machine Learning Research. 2015. No 16. P. 2023--2049.
\bibitem{Lopez2016}
	\textit{Lopez-Paz D., Bottou L., Scholkopf B., Vapnik V.} Unifying Distillation and Privileged Information // In International Conference on Learning Representations. Puerto Rico, 2016.
	
\bibitem{sutskever2014}
	\textit{Sutskever I., Vinyals O., Le Q.} Sequence to Sequence Learning with Neural Networks~// Advances in Neural Information Processing Systems, 2014. Vol.~2. P.~3104--3112.
	
\bibitem{Chunyan2016}
	\textit{Li C., Chen C., Carlson D., Carin L.} Preconditioned Stochastic Gradient Langevin Dynamics for Deep Neural Networks~// Thirtieth AAAI Conference on Artificial Intelligence.~---~Phoenix, USA, 2016. P.~1788--1794.
	
\bibitem{Tibshirani1996}
	\textit{Tibshirani R.} Regression shrinkage and selection via the Lasso~// Journal of the Royal Statistical Society, 1996. Vol.~58. P.~267--288.
	
\bibitem{Hastie2005}
	\textit{Zou H., Hastie T.} Regularization and variable selection via the Elastic Net~// Journal of the Royal Statistical Society, 2005. Vol.~67. P.~301--320.
	
\bibitem{srivastava2014}
	\textit{Srivastava N., Hinton G., Krizhevsky A., Sutskever I., Salakhutdinov R.} Dropout: A Simple Way to Prevent Neural Networks from Overfitting~// Journal of Machine Learning Research, 2014. Vol.~15. P.~1929--1958.
	
\bibitem{molchanov2017}
	\textit{Molchanov D., Ashukha A., Vetrov D.} Variational Dropout Sparsifies Deep Neural Networks~// 34th International Conference on Machine Learning.~---~Sydney, Australia, 2017. Vol.~70. P.~2498--2507.
	
\bibitem{cun1990}
	\textit{LeCun Y., Denker J., Solla S.} Optimal Brain Damage~// Advances in Neural Information Processing Systems, 1989. Vol.~2. P.~598--605.
	
\bibitem{grabovoy2019}
	\textit{Грабовой А. В., Бахтеев О. Ю., Стрижов В. В.} Определение релевантности параметров нейросети~// Информатика и ее применения, 2019. Т.~13. Вып.~2. С.~62--70.
\bibitem{grabovoy2020}
	\textit{Грабовой А. В., Бахтеев О. Ю., Стрижов В. В.} Введение отношения порядка на множестве параметров аппроксимирующих моделей~// Информатика и ее применения, 2019. Т.~14. Вып.~2. С.~58--65.

\bibitem{Mandt2017}
	\textit{Mandt S., Hoffman M., Blei D.} Stochastic Gradient Descent as Approximate Bayesian Inference~// Journal Of Machine Learning Research, 2017. Vol.~18. P.~1--35.
	
\bibitem{Kingma2014}
	\textit{Kingma D., Ba L.} Adam: A Method for Stochastic Optimization~// 3rd International Conference on Learning Representations.~---~San Diego, USA, 2015.

\bibitem{Boston}
	\textit{Harrison D.,  Rubinfeld D.} Hedonic prices and the demand for clean air~// Journal of Environmental Economics and Management, 1991. Vol.~5. P.~81--102.

\bibitem{mnist}
	\textit{LeCun Y.,  Cortes C., Burges C.} The MNIST dataset of handwritten digits, 1998. \url{http://yann.lecun.com/exdb/mnist/index.html}

\bibitem{maclarin2015}
	\textit{Maclaurin D.,  Duvenaud D., Adams R.} Gradient-based Hyperparameter Optimization Through Reversible Learning~// Proceedings of the 32th International Conference on Machine Learning, 2015. Vol.~37. P.~2113--2122.
\bibitem{luketina2015}
	\textit{Luketina J.,  Berglund M., Raiko T., Greff K.} Scalable Gradient-based Tuning of Continuous Regularization Hyperparameters~// Proceedings of the 33th International Conference on Machine Learning, 2016. Vol.~48. P.~2952--2960.
\bibitem{bishop2006}
	\textit{Bishop C.} Pattern Recognition and Machine Learning, 2006. Pp.~396.
\bibitem{neychev2016}
	\textit{Neychev R.,  Katrutsa A., Strijov V.} Robust selection of multicollinear features in forecasting~// Factory Laboratory, 2016. Vol.~82. P.~68--74.
\bibitem{cun1990}
	\textit{LeCun Y.,  Denker J., Solla S.} Optimal Brain Damage~// Advances in Neural Information Processing Systems, 1989. P.~598--605.
\bibitem{molchanov2017}
	\textit{Molchanov D.,  Ashukha A., Vetrov D.} Variational Dropout Sparsifies Deep Neural Networks~// Proceedings of the 34th International Conference on Machine Learning, 2017. Vol.~70. P.~2498--2507.
\bibitem{neal1995}
	\textit{Neal A.,  Radford M.} Bayesian Learning for Neural Networks, 1995.
\bibitem{sutskever2014}
	\textit{Sutskever I.,  Vinyals O., Le Q.} Sequence to Sequence Learning with Neural Networks, 2014. Vol.~2. P.~3104--3112.
\bibitem{graves2011}
	\textit{Graves A.} Practical Variational Inference for Neural Networks, 2011. P.~2348--2356.
\bibitem{louizos2017}
	\textit{Louizos C., Ullrich K., Welling M.} Bayesian Compression for Deep Learning, 2017. P.~3288--3298.

\end{thebibliography}

% Приложения
%\input{./appendices.tex}

\end{document}
